\documentclass[conference]{IEEEtran}
\IEEEoverridecommandlockouts
% The preceding line is only needed to identify funding in the first footnote. If that is unneeded, please comment it out.
\usepackage{cite}
\usepackage{amsmath,amssymb,amsfonts}
\usepackage{algorithmic}
\usepackage{graphicx}
\usepackage{textcomp}
\usepackage{xcolor}
\usepackage{verbatim}
\usepackage{hyperref}
\usepackage{array}
\def\BibTeX{{\rm B\kern-.05em{\sc i\kern-.025em b}\kern-.08em
    T\kern-.1667em\lower.7ex\hbox{E}\kern-.125emX}}
\begin{document}
\renewcommand{\figurename}{Figure}


\title{Breaking the Linear Barrier: A Multi-Modal LLM-Based System for Navigating Complex Web Content}

\author{\IEEEauthorblockN{Gabriel Moterani}
\IEEEauthorblockA{\textit{Digital Innovation Lab} \\
\textit{Faculty of Computer Science and Technology} \\
\textit{Algoma University}\\
Brampton, Canada \\
gamorimmoterani@algomau.ca}
\and
\IEEEauthorblockN{Wenjun (Randy) Lin}
\IEEEauthorblockA{\textit{Digital Innovation Lab} \\
\textit{Faculty of Computer Science and Technology} \\
\textit{Algoma University}\\
Brampton, Canada \\
randy.lin@algomau.ca}
}

\maketitle

\begin{abstract}
Visually impaired users still face fundamental obstacles when interacting with complex, dynamic websites. Conventional screen readers expose pages in a strict linear order, offer little semantic context for visual media, and provide limited context regarding the page content. This paper introduces a multi-modal accessibility framework combining Large Language Models (LLMs), Computer Vision, and dynamic DOM manipulation to significantly enhance semantic clarity, non-linear navigation, and interaction richness. By interpreting visual and textual web content contextually and adapting it into an intuitive, conversationally navigable interface, our method provides a foundation for visually impaired users to interact effectively with previously inaccessible or challenging digital experiences.

The deployment of a functional prototype on a modern web browser illustrates the capability of the proposed system to interact with diverse websites and tasks. The research team selected Canada's most frequented websites to assess the system's efficacy in enhancing contextual understanding of the page content and enabling navigation through pages and actions via a chat-driven interface. A comprehensive demonstration was executed using a prominent ticketing site, which facilitated users in obtaining a deeper understanding of the page while guiding them towards the successful purchase of concert tickets.By illustrating how vision–language reasoning can be coupled with low‑level browser control, this work lays the groundwork for future efforts in performance optimization, large‑scale evaluation, and personalization across diverse web contexts.
\end{abstract}

\begin{IEEEkeywords}
Web Accessibility, LLM, Accessibility DOM, Human-Computer Interaction, Context-Aware Systems
\end{IEEEkeywords}

\section{Introduction}\label{intro}

Web accessibility has posed a significant challenge to developers since the inception of web browsers capable of rendering HTML markup into visual pages. The evolution of technology has led to the development of systems such as screen readers, which convert the parsed content into audio descriptions. These descriptions are derived from the interpretation of the website's code. With the increasing adoption of the internet and advancements in modern browser capabilities, innovative techniques have been devised, enabling developers to produce code that enhances both screen-reader capabilities and navigation for users with visual impairments. The World Wide Web Consortium (W3C) was established as a consortium to define standards for the advanced implementation of the internet. While addressing multiple concerns, the organization has proposed standards that are continually evolving to enhance the user experience for individuals with impairments. The Web Content Accessibility Guidelines (WCAG), established in May 1999, provide the foundational guidelines that developers should integrate into their code to improve communication with accessibility tools. Nevertheless, although these guidelines hold the potential to enhance user experience, studies indicate a low (or poor) adoption of these practices. \cite{abuaddous2016web, antonelli2018survey} 

These challenges persist even as awareness and technological capabilities have advanced. Notably, 88\% of websites on the World Wide Web are still not accessible, reflecting a persistent gap between inclusive design principles and their real-world application \cite{webaccess2024, martins2024large}.

Despite the existence of the technologies to convert text into audio and to provide a good navigation experience for impaired users, the reality demonstrates that navigating web could be still a challenge for the avarege impaired person. Another reason for this disconnect lies in the limitations of traditional assistive tools such as screen readers. These systems rely on linear textual content interpretation, which struggles with the complexity and interactivity of modern web experiences—particularly on platforms that depend on multimedia content, multi-step forms, or dynamically updated dashboards. Compounding the issue is the inconsistent implementation of accessibility standards such as the WCAG, most recently updated with WCAG 3.0 in 2024. While these guidelines offer a robust framework for accessible design, their application often falls short due to a lack of developer expertise, improper use of ARIA (Accessible Rich Internet Applications) tags, and a prevailing focus on visual aesthetics rather than semantic clarity \cite{gbd2021, wcagchallenges2025}.

Recent advancements in context-aware technologies, including Large Language Models (LLM), Computer Vision, and Machine Learning, present new opportunities for narrowing the accessibility gap. These technologies can augment traditional assistive devices by functioning as intelligent agents, capable of real-time interpretation, restructuring, and interaction with web content. They facilitate the creation of alternative contexts within the accessibility framework, enabling effective content rendering in audio readers and fostering new modes of internet interaction for users with various impairment issues, such as motor disabilities. Such systems can dynamically adapt interfaces to align with users' goals, offering not only improved content understanding but also richer forms of interaction.

In this study, we introduce a browser-based accessibility framework driven by a multi-modal architecture that synergizes vision-language reasoning, contextual manipulation, and dialogue-based task planning. Our framework is realized as an extension compatible with prominent web browsers such as Google Chrome, Safari, and Firefox, on both desktop and mobile platforms. Furthermore, it augments the structural and semantic aspects of web content. This system operates collaboratively across the Document Object Model (DOM) and the browser's Accessibility Tree, equipping visually impaired users with enhanced capabilities for more efficient browsing, comprehension, and interaction with web content.

The remainder of this paper is organized as follows: \autoref{review} reviews related work and accessibility technologies; \autoref{system} details the proposed system and its modular architecture; \autoref{methodology} describes the implementation methodology; \autoref{demo} presents a practical usage scenario demonstrating the system in action; and \autoref{conclusion} concludes with findings, limitations, and directions for future research.


\section{Literature Review}\label{review}

Multiple studies over the past two decades have documented how screen readers process web pages strictly in a linear sequence, forcing visually‑impaired users to traverse headings, menus, and decorative elements before reaching relevant content. \cite{Ramakrishnan2017-rn, wcagchallenges2025, martins2024large, abuaddous2016web} Large‑scale audits continue to report widespread keyboard‑navigation barriers, even on sites that advertise compliance with success criteria in WCAG 2.x and the 2024 WCAG 3.0 draft \cite{wcagchallenges2025}. 

Empirical user observations show that the absence of semantic grouping leads to excessive keystrokes and cognitive overload when interacting with dashboards, multi‑column news layouts, and e‑commerce grids. Despite incremental improvements in screen‑reader software, the underlying linear traversal model remains unchanged—leaving the fundamental navigational burden on the user. \cite{ferdous2021semantic}

When browsing the internet, users without impairments can efficiently skim through and scan texts, thereby saving time and disregarding irrelevant information. In contrast, individuals who depend on screen-readers and other accessibility tools are deprived of this ability. Consequently, such users must often listen to a substantial portion of content to determine whether to continue seeking information on a particular website. \cite{Ramakrishnan2017-rn} This situation underscores that, despite the increased adoption of relevant guidelines and technological advancements, a more contextual and dynamic approach should be pursued.

Accessible design guidelines require developers to supply meaningful alternative text, ARIA roles, and programmatic relationships, yet field surveys reveal that these elements are either missing or optimized for search‑engine ranking rather than human comprehension \cite{wcag2023}. Where alt text is present, it is often terse ("image1") or redundant with surrounding captions, providing little help in grasping complex visuals such as charts, infographics, or hero banners. Even on sites that partially implement WCAG techniques, dynamic content injected by third‑party ad networks or CMS templates escapes validation, breaking reading order and semantic context \cite{wcagchallenges2025}. As a result, visually‑impaired users must infer meaning from fragments that were never authored with their experience in mind.

Recent prototypes illustrate how LLM, computer vision, and DOM automation can enrich accessibility beyond traditional screen readers. Commercial and research extensions now parse shopping pages to surface product attributes in a simplified view, reducing navigation steps \cite{prakash2024}. Task‑oriented agents powered by LLMs translate natural‑language commands into browser actions, demonstrating promise in filling out forms, filtering tables, or performing multi‑step tasks  \cite{he2024webvoyager}. While these systems validate the feasibility of AI‑mediated assistance, they remain domain‑specific, rely on brittle heuristics, or lack an integrated mechanism for maintaining conversational context across page transitions.\cite{kodandaram2024,mehendale2024}

Principal investigations are concentrated on employing a visual compiler methodology, wherein the context derived from 'screenshots' serves as the primary element for parsing page content, facilitating alterations to the webpage, or enabling user navigation with the support of auxiliary LLM tools. Although screenshots offer a comprehensive understanding of the page's context, LLMs interact with textual data, establishing a paradigm in which their application is not optimized for textual analysis. The primary applications proposed herein is deficient in its capacity to integrate seamlessly into the conventional workflow of an internet user. As an independent application, while it addresses certain aspects of the problem, it simultaneously engenders a new challenge by eschewing the utilization of browsers as its principal mechanism. \cite{he2024webvoyager, mehendale2024, prakash2024}

\subsection*{Synthesis and Identified Gaps}

Taken together, the literature confirms that technical progress has not yet resolved several critical barriers:

\begin{enumerate}
\item \textbf{Non‑linear navigation is still unsupported.} Screen readers continue to expose content in a purely sequential manner, forcing users through irrelevant elements before they reach task‑relevant information \cite{Babu_2013}.  
\item \textbf{Semantic context for visual content is insufficient.} Alt text and ARIA labels are inconsistently authored, providing at best superficial descriptions that fail to convey rich visual meaning \cite{Chintalapati_2022}.  
\item \textbf{General‑purpose, task‑driven interaction remains elusive.} AI‑augmented tools demonstrate success in narrow domains but do not yet offer a unified solution for complex, cross‑site workflows \cite{prakash2024,kodandaram2024,mehendale2024}.  
\end{enumerate}

Addressing these gaps requires a comprehensive approach that builds upon existing assistive technologies while introducing innovative solutions. Our work focuses on developing a system that leverages the widespread adoption of screen readers and browser extensions among Blind and Vision Loss (BLV) users. By combining established accessibility tools with advanced AI capabilities, we aim to create a more adaptive and intuitive navigation experience.Our approach is further distinguished by employing browsers as the primary framework for developing functionalities that seamlessly integrate with the existing workflow of users, thereby facilitating adoption and enhancing the ease of website interaction. 

This methodology integrates multiple layers of assistance: enhancing semantic understanding through improved content interpretation, providing context-aware navigation options, and enabling task-oriented interactions through natural language processing. The resulting system seeks to bridge the gap between traditional accessibility tools and modern web interfaces, offering BLV users a more seamless and efficient way to interact with complex web content.

\section{Proposed System Overview}\label{system}

While browsing the internet, users rely on software applications capable of facilitating data exchange with cloud servers via the Hypertext Transfer Protocol (HTTP), ultimately acquiring Hypertext content formatted in Hypertext Markup Language (HTML). These applications convert the textual representation of web pages into a visual format. These software applications are denominated as web browsers. At present, Google Chrome constitutes the primary browser utilized within desktop environments, accounting for a 66\% market share. Conversely, Safari is predominantly used on mobile platforms employed by 22\%  of mobile users\cite{browserstats2025}.

Web browsers have evolved to facilitate applications that function in conjunction with the product through the utilization of browser APIs. These applications, known as extensions, are installable within browsers and serve to augment the overall capabilities of the browser. Extensions enable developers to create enhancements for standard navigation, facilitating features such as text grammar correction, ad blockers, and other mechanisms. In this project, extensions are employed to interact with the DOM, which is the technical representation of elements on a web page in browsers, as well as the Accessibility Tree, which shares the same representation but is focused on what screen readers (SR) mechanisms utilize to convert content for visually impaired users.

As shown in \autoref{tab:architecture}, our architecture orchestrates four cooperative modules around a shared task graph and state log. The LLM serves as the reasoning core, translating between raw pixels/DOM, natural‑language dialog, and low‑level interaction events.

\subsection{Visual‑Layout Interpreter (VLI)}

The VLI system acquires periodic snapshots of Document Object Model (DOM) elements, analyzing the content through various methodologies to construct a JavaScript Object Notation (JSON) representation of a website's elements. This JSON representation undergoes further analysis to extract images, textual content, interactive actions, advertisements, navigational content, and also to identify WCAG infringements. In essence, the VLI is tasked with the collection and conversion of webpage elements into a technical format that facilitates modification and interaction by other modules within the application. A simplification of this module can be inferred in the \autoref{fig:vli}.

\begin{figure}[h]
\centering
\includegraphics[width=\columnwidth]{images/VLI.png}
\caption{Simplification of VLI module.}
\label{fig:vli}
\end{figure}


\subsection{Contextual Modification Engine (CME)}

As demonstrated in \autoref{fig:cme}. The CME works as the an orchquestrator for the aplication, it consumes, text, images, actions and wcag infrigiments created by the VLI to use multiple techniques to contextually parse the content, acting as the main engine for the other modules. After receiving data the CME will interact with the backend to receive a summary of the page, and to require for each image provided that has not an adequate alternative text a new content improving the context of the overall pages. Summary of the page is also provided for the user in the top of the page to allow user to skim and make a decision if the user wants to pursuing interacting with that webpage. Images alt properties are inserted async providing user the best content without taking time to process all content. Finally the WCAG infrigiments are parsed by the LLM model trying to find adequate methods to fix the tags to be compliance with the standards and guidelines providing a better experience.

\begin{figure}[h]
\centering
\includegraphics[width=\columnwidth]{images/CME.pdf}
\caption{Simplification of CME module.}
\label{fig:cme}
\end{figure}

\subsection{Conversational Plan Manager (CPM)}

The CPM runs a mixed‑initiative chat: it elicits the user's intent, iterates over the VLI exported actions JSON, and creates reasoning using a LLM model to create a output of actions to be taken to complete each proposed task. The LLM is responsible to understaning each possible action is responsible for each task as well as what JS command should be triggered to make execute that command. That plan is then used by the Multimodal Interaction Module to execute commands and then compelete tasks. A representation of that communication with the backend to acquire the reasoing tasks can be veriefied in \autoref{fig:communication}.

\subsection{Multimodal Interaction Executor (MIE)}

The MIE methodically progresses through the plan, rigorously evaluating the suggested JS commands, confirming the presence of each proposed interactive element, making necessary adjustments, and ultimately executing the JS commands required to achieve the specified user task.

\begin{table}[h]
\centering
\caption{System Components and Their Roles}
\label{tab:architecture}
\small
\renewcommand{\arraystretch}{1.3}
\begin{tabular}{|>{\raggedright\arraybackslash}p{0.8cm}|>{\raggedright\arraybackslash}p{2.5cm}|>{\raggedright\arraybackslash}p{4cm}|}
\hline
\textbf{Name} & \textbf{Functional Role} & \textbf{Rationale} \\
\hline
\texttt{VLI} & Visual understanding of the current page & Parses the rendered page DOM into a structured semantic graph usable by language and planning components. \\
\hline
\texttt{CME} & Parses content and provide aditional context & Communicates with Backend to request aditional content modifiying page for a better experience. \\
\hline
\texttt{CPM} & Dialogic intent capture \& planning & Conducts a mixed-initiative dialog to clarify the user's objective, decomposes it into an executable plan, and maintains a progress log. \\
\hline
\texttt{MIE} & Autonomous action execution & Carries out the CPM's plan (clicks, typing, drag-and-drop, drawing), and after each step re-invokes the VLI to verify success or recover. \\
\hline
\end{tabular}
\end{table}


\subsection*{Example Workflow}

During the system's development, multiple visually impaired individuals were consulted regarding challenging tasks encountered on the internet. A prevalent issue identified was the navigation of online news sites. Despite developers' attempts to enhance accessibility, the dynamic elements, such as advertisements and diverse news services, render navigation overwhelming and nearly unmanageable. The transient nature of news, coupled with the vast volume of content produced, significantly impedes content adaptation for visually impaired users. 

In this scenario, we are modeling the basic process of accessing a news website to obtain insights regarding the essential conclusions of the most significant articles, as demonstrated in \autoref{tab:sequence}.

\vfill
\begin{table}[h]
\centering
\caption{Phase and Module Sequence}
\label{tab:sequence}
\footnotesize
\renewcommand{\arraystretch}{2}
\begin{tabular}{|c|p{5.5cm}|}
\hline
\textbf{Phase} & \textbf{Module Sequence} \\
\hline
Intent capture & CPM → user: "What would you like to do?" → user: "Read today's headlines." \\
\hline
Context analysis & VLI parses home page; CME adapts headline list. \\
\hline
Plan and Exection & CPM drafts plan, while MIE executes. User navigates to article pages where the VLI presents summary and loop continues\\
\hline
\end{tabular}
\vspace{0.5cm}
\end{table}

\section{Prototype Architecture}\label{methodology}

The proposed system is demonstrated through a prototype system to illustrate navigational capabilities for individuals with visual impairments. This architecture can be methodically categorized into two primary structures: Front-end Development and Back-end Development, as shown in \autoref{fig:communication}.

The Front-end architecture facilitates the interaction between the browser and the tool, involving the extraction of data from the page, its transformation into pertinent nodes, and the establishment of interactions between the system-page and the user-system.

\begin{figure}[h]
\centering
\includegraphics[width=\columnwidth]{images/5.pdf}
\caption{Simplification of the Backend - Frontend Communication.}
\label{fig:communication}
\end{figure}

\subsection{Front-end Structure}

In the process of constructing the Front-end architecture, a comprehensive investigation into frameworks was undertaken to identify those that would offer enhanced performance and maintainability for the tool, as well as competence in interfacing with browser APIs effectively. The Plasmo framework for web extension development was selected as the foundational technology for the extension's creation. Plasmo integrates the web-ext library to facilitate the development of extensions using Javascript and Typescript languages, enabling the incorporation of sophisticated frontend libraries such as Reactjs. The amalgamation of these technologies empowered the developers to craft efficient, streamlined, and consistent interactions among pages, systems, and users.

Upon the rendering of a new page, the Front-end system detects alterations in the primary DOM elements, recognizing a modification and thereby activating the designated system to acquire both the DOM and Accessibility trees. This process involves filtering out superfluous content utilizing ad blocking technology and a readability library, a JavaScript library specifically designed to enhance readability by removing unnecessary content.

Since the technologies underpinning LLM depend on the volume of input text, which is converted into tokens — the fundamental units of LLM data — it is crucial that the frontend system sanitizes the input to supply the data interpreter with only essential information. Failure to do so necessitates that the content be divided into smaller segments for processing. This segmentation can result in increased rendering time of contextual elements, elevated server costs, and potential loss of context.

To generate a cleaner version of the DOM, an initial pre-processing step was used to exclude advertising-related elements. This was achieved through the application of a set of community-maintained blacklist rules, implemented via a custom reimplementation of a DOM-cleaning library designed to apply these filtering criteria. Following this sanitization phase, the Readability JS library—originally developed by Mozilla for Firefox—was utilized to further refine the DOM by removing non-essential nodes, thereby preserving only the content most relevant to user navigation and comprehension.

Subsequent to the purification of elements, the front-end will extract the main content transforming into a JSON representation, which is more iterable over programming languages than the pure DOM representation. This JSON is iterated over to extract main components that will be the input of each module described in the previous sections. A new representation is created with the actions, texts, images and tags with WCAG violations.

Finally front-end establish communication with the back-end modules through fundamental HTTPS requests using the fetch library. These modules will furnish the requisite context and information to be elaborated upon in the back-end subsection of this section. The front-end will then transform this contextual information into DOM modifications aimed directly at engaging with the Accessibility Tree, thereby enhancing the end-user experience. Furthermore, the extension will inject HTML elements into any web page, facilitating user interaction with the extension system, thereby enabling effective user-system communication and task execution by the system on behalf of the user.

\subsection{Back-end Structure}

The backend system is comprised of a dockerized Flask Python server configured to receive messages on a web accessible endpoint. These messages are processed and converted into executable tasks for the system, resulting in outputs that are subsequently utilized by the Front-end in response to the queried messages. 

Although the backend encompasses a variety of techniques, it is essential to articulate four principal modules to provide clarity regarding its functionality.

\subsubsection{Visual Computation Methodology}

The backend component is tasked with receiving all visual elements, which may include binaries, image URLs, CSS (Cascading Style Sheets) cascade elements, or any other elements that pertain exclusively to a visual context. Based on the specific format of each element, the system processes these into images that are forwarded to a Computer Vision Machine Learning System. This system employs an algorithm that has been trained with an extensive dataset comprising billions of images, enabling it to accurately identify and describe the content within the images. The resulting image description is synthesized with a summary generated by an AI (Artificial Intelligence) language model, along with the proximate elements, to deliver a comprehensive contextual description of the visual element.

\subsubsection{Context Aware Methodology}

This component is responsible for receiving text, processed and queried by the Front-end from the DOM. This text is transmitted in JSON format and encompasses both the page content and the type of HTML tag encapsulating the text. The server then processes this information using a Large Language Model system, which extracts both the summary and additional context from the page. This refined information is subsequently utilized across various Back-end and Front-end modules throughout the system's operational workflow.

\subsubsection{WCAG Compliance Methodology}

Component receives a JSON representation of DOM elements that violates WCAGs rules, upon which it processes this data through a finely tuned Large Language Model system. This LLM has been trained using data from websites exemplary of both effective and ineffective adherence to WCAG guidelines. Consequently, the LLM returns a JSON containing element identifiers and suggestions for compliance modifications or new tags. These alterations facilitate the conversion of the website's accessibility framework into a format that can be accurately interpreted by TTS systems. The front-end is then responsible for elucidating this information and accordingly transforming the page.

\subsubsection{Conversational Task Executer Methodology}

Following the primary execution phase for context preparation, the server promptly stores an available action representation of the visited website. This object facilitates rapid and efficient communication with the AI, which is powered by an LLM. The content is queried via LLMs, enabling the identification of specific portions of the HTML code necessary for various tasks solicited by users. Afterward, it returns executable actions via JavaScript. The front-end, equipped with this information, utilizes JavaScript commands to perform tasks on behalf of visually impaired users.

\section{Prototype Demonstration}\label{demo}

The presented example workflow aims to elucidate a straightforward task that can be performed by any user seeking to obtain fundamental information regarding the acquisition of tickets to a concert. In this prototype demonstration, we will simulate the challenges faced by a visually impaired individual in comprehending the available concerts on a relevant ticket website. Following the installation of the Google Chrome extension, the DOM elements will undergo modification, resulting in alterations to the accessibility tree, as illustrated in \autoref{fig:start}.

\begin{figure}[h]
\centering
\includegraphics[width=\columnwidth]{images/1.png}
\caption{Page is loaded and system start parsing content}
\label{fig:start}
\end{figure}


Humans possess the ability to scan and skim web pages, allowing them to assess the relevance of the page and the accuracy of the information prior to engaging with the full content. This seemingly straightforward human skill is often absent among BLV users. When a website is accessed using the extension, the initial task for the system is to generate a summary of the page. This summary facilitates the user's decision-making process regarding the potential navigation of the site. Ideally, the summary should be prepared within the standard Large Contentful Paint (LCP) average time of three seconds. This enables users to make an informed choice about engaging with the page and provides the interface necessary for interaction.

\begin{figure}[h]
\centering
\includegraphics[width=\columnwidth]{images/2.png}
\caption{Summary and Chat Interfaces displayed}
\label{fig:chat}
\end{figure}

\autoref{fig:chat} provides a visual demonstration of the elements inserted in the accessibility tree to be used by screen readers to deliver the new content to BLV user. After the manipulation of the DOM elements into a summary and a chat interface the system continue asynchronously to work in the effort to manipulate HTML to be WCAG compliance.

This process employs the axe-core JavaScript library to identify tags and images that do not comply with WCAG standards. The backend system then iterates over these elements to generate the missing attributes or enhance attributes that remain non-compliant. Following communication with the backend, the frontend proceeds to adjust the webpage to ensure the end user can navigate effectively, notwithstanding the incomplete implementation of accessibility features.

\begin{figure}[h]
\centering
\includegraphics[width=\columnwidth]{images/3.png}
\caption{Image and it's alternative text created by LLM}
\label{fig:alt}
\end{figure}

\autoref{fig:alt} illustrates an image extracted from the website, for which the extension retrieved a novel alternative description that would otherwise be inaccessible to the BLV user, rendering it invisible to them. Such content has the potential to enhance the user's navigational experience if the end user deems it beneficial.

User can now interact with the page either using it's own navigation or by chating with the tool, prompting the system to perform actions on it's behalf. In this demonstration user prompted the AI to buy tickets to Bryan Adams concert. After analyzing the DOM elements the accessibility tool then clicked in the link (using JS actions) related to that concern, navigating user to a new page where the journey repeats itself.

\begin{figure}[h]
\centering
\includegraphics[width=\columnwidth]{images/4.png}
\caption{Page navigated after the AI prompted}
\label{fig:show}
\end{figure}

As shown in \autoref{fig:show}, the system successfully facilitated user navigation to a new page, where now users need to select seats and perform payment. It is imperative to consider that in subsequent versions of this application, the CPM should be capable of continue executing tasks on new pages to ensure the continual achievement of the targeted user objective. Due the time constrains the team yet not developed the continuously loop over tasks and pages. Being part of the scope for future implementations.


\section{Findings}

As illustrated in prior research, the principal challenge associated with these technologies is establishing their efficacy across diverse web page contexts \cite{prakash2024, he2024webvoyager}. To analyze and verify the application's efficacy, the authors opted to apply the extension to the ten most accessible websites in Canada as of 2025. Additionally, several websites not ranked among the most visited were included to provide a broader context. The list of websites are presented in \autoref{tab:websites}

\begin{table}[h]
\centering
\caption{Websites Analyzed}
\label{tab:websites}
\small
\renewcommand{\arraystretch}{1.3}
\begin{tabular}{|>{\raggedright\arraybackslash}p{2.5cm}|>{\raggedright\arraybackslash}p{4cm}|>{\raggedright\arraybackslash}p{2.5cm}|>{\raggedright\arraybackslash}p{2.5cm}|}
\hline
\textbf{Website} & \textbf{Short Description} & \textbf{Category} & \textbf{Monthly Access} \\
\hline
Wikipedia & Articles and Informations about any subject & Information & 1.36M \\
\hline
Amazon.ca & Multi items online shopping store & Shopping & 153M \\
\hline
CBC & News information about Canada & News & 2M \\
\hline
Yahoo & Services and News of Yahoo Network & News & 235M \\
\hline
NHL & Hockey Information and Merchandise & Sports & 11M \\
\hline
Realtor.ca & Realtor app & Services & 34M \\
\hline
Weather.com & Weather Information & Information & 769M \\
\hline
Apple.com & Technology Hardware & Shopping & 188M \\
\hline
Indeed.com & Job posting website & Services & 24M \\
\hline
BMO & Bank of Montreal website & Banking & 300K \\
\hline
\end{tabular}
\end{table}

All websites were accessed 5 times, collecting the following data:
\begin{itemize}
    \item Html Size
    \item Html Size after cleanup 
    \item JSON Chars Size
    \item Number of Images
    \item Number of Images that don't have any alt tag
    \item Number of images which alt tag char count is more than 240
    \item Number of possible actions
    \item WCAG violations
    \item Time to LLM provide a summary
    \item Average time to process image
    \item Number of times that the main task was completed
\end{itemize}

The efficacy of the system is calculated by the number of times the main task was completed. The main task is defined as the task that the user prompted the system to complete. The system is considered to have completed the task if the user is able to complete the task without any assistance from the system.

Eficacy is calculated by the following formula:

\begin{equation}
    Eficacy = \frac{Number of times the main task was completed}/{Number of times the main task was attempted}
\end{equation}

\begin{table}[h]
\centering
\caption{Website Analysis Results}
\label{tab:results}
\small
\renewcommand{\arraystretch}{1.3}
\begin{tabular}{|>{\raggedright\arraybackslash}p{3cm}|>{\raggedright\arraybackslash}p{2.5cm}|>{\raggedright\arraybackslash}p{2.5cm}|>{\raggedright\arraybackslash}p{2.5cm}|>{\raggedright\arraybackslash}p{2.5cm}|}
\hline
\textbf{Website} & \textbf{HTML Size in Chars} & \textbf{ Number of Images} & \textbf{WCAG Violations} & \textbf{Efficacy \%} \\
\hline
Wikipedia & 140852 & 23 & 4 & 80 \\
\hline
Amazon.ca & 948,050 & 238 & 53 & 100 \\
\hline
CBC & 224,808 & 35 & 5 & 40 \\
\hline
Yahoo & 757,533 & 17 & 16 & 0.00 \\
\hline
NHL & 422,463 & 88 & 81 & 60 \\
\hline
Realtor.ca & 75,189 & 33 & 51 & 0.20 \\
\hline
Weather.com & 1,138,290 & 8 & 23 & 100 \\
\hline
Apple.com & 302,077 & 0 & 3 & 100 \\
\hline
Indeed.com & 399,086 & 2 & 1 & 40 \\
\hline
BMO & 2,452,943 & 368 & 0 & 80 \\
\hline
\end{tabular}
\end{table}






\section{Conclusion}\label{conclusion}

This paper has addressed critical and persistent accessibility challenges experienced by BLV navigating web. Through integrating advances in LLMs, computer vision, and dynamic DOM manipulation, we introduced a multimodal accessibility system designed as a browser extension to significantly enhance web interaction experiences for visually impaired users.

Our architecture uniquely combines four core modules—VLI, CME, CPM, and MIE to collaboratively parse visual and semantic content, dynamically reorganize web page structures, conduct intent-based conversational planning, and automate complex web interactions. The prototype demonstration using the Ticketmaster's webpage highlighted the system's practical capability to simplify complex navigation tasks, enhance semantic comprehension through improved alt text and ARIA labels, and successfully execute user-driven tasks such as information retrieval and ticket purchases.

However, despite its promising outcomes, several practical and technical limitations have been identified. First, the computational costs associated with real-time inference of LLM and computer vision models remain high, necessitating server resources equipped with high-performance GPUs (Graphics Processing Unit). Second, latency introduced by inter-module communication and iterative page rendering may affect usability during extensive interactions. Third, the prototype's ability to generalize its effectiveness across diverse web structures and complex dynamic websites requires further development and validation.

Future research directions are clear and critical for achieving broader adoption and usability. Optimizing system performance and reducing computational demands through more efficient inference methods and specialized LLM fine-tuning specifically for accessibility tasks are immediate priorities. The research and develop also aims into implementing features like continuously task executer, multi-language contexts and video parser.

Ultimately, while technical innovation forms the core of our approach, the broader vision remains creating a universally accessible internet. Emphasizing user-centered iterative development, ongoing collaboration, and inclusive design principles will ensure that advancements in AI technology translate meaningfully into equitable digital experiences for all users.


\section{Use of AI Disclosure}\label{ai-disclosure}
No artificial intelligence tools were employed in the generation of the content presented in this document. Throughout the drafting process, the Overleaf AI Assistant was utilized to review grammatical accuracy and to offer feedback on writing style and consistency, thereby suggesting improvements to the text and making modifications to the content. During the coding phase of the project, Claude AI, version 3.5, was employed to assist developers in coding tasks and error resolution.


\section{Acknowledgments}\label{ai-disclosure}
The authors extend their sincere gratitude to the members of the Algoma University Digital Innovation Lab and the anonymous reviewers for their insightful feedback on the initial drafts of this paper. Additionally, the contributions of Lucas Radaelli and Nadya Feijo, as representatives of the visually impaired community, are duly acknowledged.

% References
\bibliographystyle{unsrt}
\bibliography{references}
\end{document}
